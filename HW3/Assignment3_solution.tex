%%%%%%%%%%%%%%%%%%%%%%%%%%%%%%%%%%%%%%%%%
% Structured General Purpose Assignment
% LaTeX Template
%
% This template has been downloaded from:
% http://www.latextemplates.com
%
% Original author:
% Ted Pavlic (http://www.tedpavlic.com)
%
% Note:
% The \lipsum[#] commands throughout this template generate dummy text
% to fill the template out. These commands should all be removed when 
% writing assignment content.
%
%%%%%%%%%%%%%%%%%%%%%%%%%%%%%%%%%%%%%%%%%

%----------------------------------------------------------------------------------------
%       PACKAGES AND OTHER DOCUMENT CONFIGURATIONS
%----------------------------------------------------------------------------------------

\documentclass{article}

\usepackage{fancyhdr} % Required for custom headers
\usepackage{extramarks} % Required for headers and footers
\usepackage{graphicx} % Required to insert images
\usepackage{enumerate}
\usepackage{amsmath}
\usepackage{comment}
\usepackage{bbold}

% Margins
\topmargin=-0.45in
\evensidemargin=0in
\oddsidemargin=0in
\textwidth=6.5in
\textheight=9.0in
\headsep=0.25in 

\linespread{1.1} % Line spacing

% Set up the header and footer
\pagestyle{fancy}
\lhead{Linear Algebra with Application\\
to Engineering Computation}
\chead{}
\rhead{CME 200/ME300A\\
M. Gerritsen\\
Fall 2013}
\headheight = 40pt

\renewcommand\headrulewidth{0.4pt} % Size of the header rule
\renewcommand\footrulewidth{0.4pt} % Size of the footer rule

\setlength\parindent{0pt} % Removes all indentation from paragraphs

%Add useful short-cut commands here

%1st order derivative wrt x
\newcommand{\dxone}[1]{\frac{d#1}{dx}}
%2nd order derivative wrt x
\newcommand{\dxtwo}[1]{\frac{d^2#1}{dx^2}}
%th in the exponent (e.g. when writing ith, instead use i$\eth$)
\newcommand{\eth}{^{\text{th}}}

%short-cuts for Greek letters
\newcommand{\al}{\alpha}
\newcommand{\dlt}{\delta}
\newcommand{\eps}{\epsilon}

\newcommand{\R}{\mathbb{R}}

\newcommand{\x}{\times}
%inverse
\newcommand{\inv}{^{-1}}
%cond
\newcommand{\cond}{\mathrm{cond}}
%trace
\newcommand{\trace}{\mathrm{trace}}
%rank
\newcommand{\rank}{\mathrm{rank}}

\newcommand{\twith}{\text{ with }}
\newcommand{\tand}{\text{ and }}
\newcommand{\tfor}{\text{ for }}
\newcommand{\tor}{\text{ or }}

\newcommand{\ip}{_{i+1}}
\newcommand{\im}{_{i-1}}

\newcommand{\half}{\frac{1}{2}}
\newcommand{\oneby}[1]{\frac{1}{#1}}
\newcommand{\overto}[1]{\overset{#1}{\longrightarrow}}
%----------------------------------------------------------------------------------------
%       DOCUMENT STRUCTURE COMMANDS
%       Skip this unless you know what you're doing
%----------------------------------------------------------------------------------------

% Header and footer for when a page split occurs within a problem environment
\newcommand{\enterProblemHeader}[1]{
\nobreak\extramarks{#1}{#1 continued on next page\ldots}\nobreak
\nobreak\extramarks{#1 (continued)}{#1 continued on next page\ldots}\nobreak
}

% Header and footer for when a page split occurs between problem environments
\newcommand{\exitProblemHeader}[1]{
\nobreak\extramarks{#1 (continued)}{#1 continued on next page\ldots}\nobreak
\nobreak\extramarks{#1}{}\nobreak
}

\setcounter{secnumdepth}{0} % Removes default section numbers
\newcounter{homeworkProblemCounter} % Creates a counter to keep track of the number of problems

\newcommand{\homeworkProblemName}{}
\newenvironment{homeworkProblem}[1][Problem \arabic{homeworkProblemCounter}]{ % Makes a new environment called homeworkProblem which takes 1 argument (custom name) but the default is "Problem #"
\stepcounter{homeworkProblemCounter} % Increase counter for number of problems
\renewcommand{\homeworkProblemName}{#1} % Assign \homeworkProblemName the name of the problem
\section{\homeworkProblemName} % Make a section in the document with the custom problem count
\enterProblemHeader{\homeworkProblemName} % Header and footer within the environment
}{
\exitProblemHeader{\homeworkProblemName} % Header and footer after the environment
}
\newcommand\overmat[2]{%
  \makebox[0pt][l]{$\smash{\overbrace{\phantom{%
    \begin{matrix}#2\end{matrix}}}^{\text{$#1$}}}$}#2}

\newcommand{\problemAnswer}[1]{ % Defines the problem answer command with the content as the only argument
\noindent\framebox[\columnwidth][c]{\begin{minipage}{0.98\columnwidth}#1\end{minipage}} % Makes the box around the problem answer and puts the content inside
}

\title{Assignment 3 - Solutions}
\date{Issued: October 9, 2013}
\author{Due: October 16, in class\\
No late assignments accepted}

%----------------------------------------------------------------------------------------

\begin{document}

\maketitle
\thispagestyle{fancy}

% Problem 1
\begin{homeworkProblem}
\begin{enumerate}[(a)]
%(5)
\item Find a basis for the column space and row space of matrix $A$ given by
\[ A = \begin{bmatrix}
0 & 1 & 2 & 3 & 4 \\
0 & 1 & 2 & 4 & 6 \\
0 & 0 & 0 & 1 & 2
\end{bmatrix} \]

 %(10)
\item  Construct a matrix $B$ such that the null space of $B$ is identical to the row space of $A$.
\end{enumerate}

\problemAnswer{
\begin{enumerate}[(a)]
\item  We have the following matrix $A$:
\[A = \begin{bmatrix}
0 & 1 & 2 & 3 & 4 \\
0 & 1 & 2 & 4 & 6 \\
0 & 0 & 0 & 1 & 2
\end{bmatrix}. \]
A basis for the column space of $A$ can be found in various ways. (Finding a basis for the row space is similar.) One way is to perform Gaussian Elimination (GE) on $A^T$, then find the independent rows of $A^T$ (corresponding to the independent columns of $A$).\\
Alternatively, one can also do a GE on $A$ and then use the ``subtle theorem" discussed in class. Specifically, GE gives us the matrix $U$. The independent columns in $U$ are the columns with the nonzero pivots. Recall that the ``subtle theorem" tells that the independent columns in $A$ are the columns in those same positions. This would give us the choices of columns of $A$ as a basis for the column space.\\
We can also make a dimensional argument to justify our choices of columns of $A$ as a basis: we select independent columns of $A$ and we select as many as the number of the rank of $A$, i.e. the dimension of the column space of $A$. Now we perform GE on $A$
\[
\begin{bmatrix}
0 & 1 & 2 & 3 & 4 \\
0 & 1 & 2 & 4 & 6 \\
0 & 0 & 0 & 1 & 2
\end{bmatrix} \xrightarrow{\text{row } 2 - \text{ row } 1} \begin{bmatrix}
0 & 1 & 2 & 3 & 4 \\
0 & 0 & 0 & 1 & 2 \\
0 & 0 & 0 & 1 & 2
\end{bmatrix} 
\xrightarrow{\text{row } 3 - \text{ row } 2} \begin{bmatrix}
0 & 1 & 2 & 3 & 4 \\
0 & 0 & 0 & 1 & 2 \\
0 & 0 & 0 & 0 & 0
\end{bmatrix} = U
\]
From GE, we have the upper triangular matrix $U$. Since we have 2 nonzero rows (or 2 pivots), so the rank of $A$ is 2. A basis for its column space can be its $2^{\text{nd}}$ and $4^{\text{th}}$ column:
\[ \vec{a}_1 = [1\ 1\ 0]^T\ \ \ \vec{a}_2 = [3\ 4\ 1]^T. \]
A basis for its row space can be the first and second row of $A$:
\[ \vec{r}_1^T = [0\ 1\ 2\ 3\ 4]\ \ \ \vec{r}_2^T = [0\ 1\ 2\ 4\ 6]. \]
\end{enumerate}
}

\problemAnswer{
\begin{enumerate}[(b)]
\item We would need $B\vec{r}_1 = \vec{0} \tand B\vec{r}_2 = \vec{0}$. $B$ should have five columns, so $n = 5$. Its null space has dimension 2. For any matrix, the dimension of the nullspace equals $(n-r)$, where $r$ is the rank of $B$. Here $n-r = 2 \tand n = 5$, hence $B$ must be of rank $r = 3$. We might as well make $B$ a $3 \x 5$ matrix.\\
Note that the nullspace of a matrix and the row space of that same matrix are orthogonal to each other. So here we need $B$ to be a $3\x5$ matrix with 3 independent rows that are orthogonal to $\vec{r}_1,\ \vec{r}_2$. For any row $\vec{b}^T$ of matrix $B$, we must have
\[ \begin{bmatrix}
\vec{r}_1^T \\ \vec{r}_2^T
\end{bmatrix} \vec{b} = \vec{0}, \quad \text{ so } \begin{bmatrix}
0 & 1 & 2 & 3 & 4 \\
0 & 1 & 2 & 4 & 6 \end{bmatrix} \vec{b} = \vec{0}\]
We have to find the solution $\vec{b}$ of the above equation. These give two equations for 5 unknowns. As $\vec{r}_1 \tand \vec{r}_2$ are independent, we have $5-2 = 3$ degrees of freedom. In order to find the solution of $\begin{bmatrix}
\vec{r}_1^T \\ \vec{r}_2^T
\end{bmatrix} \vec{b} = \vec{0}$, we perform GE on matrix $\begin{bmatrix}
\vec{r}_1^T \\ \vec{r}_2^T
\end{bmatrix}$, 
\[\begin{bmatrix}
0 & 1 & 2 & 3 & 4 \\
0 & 1 & 2 & 4 & 6 \end{bmatrix} \xrightarrow{\text{row } 2 - \text{ row } 1} \begin{bmatrix}
0 & 1 & 2 & 3 & 4 \\
0 & 0 & 0 & 1 & 2 \end{bmatrix}
\]
Let the solution $\vec{b} = [b_1 \ b_2 \ b_3 \ b_4 \ b_5]^T$. Then we have $$b_4 + 2b_5 = 0, \tand b_2 + 2b_3 + 3b_4 + 4b_5 = 0$$ The first equation gives $b_4 = -2b_5$. Substituting this into the second equation, we have $b_2 + 2b_3 - 2b_5 = 0$. So, $b_2 = -2b_3 + 2b_5$. Therefore, we have the solution in the form $$\vec{b} = \begin{bmatrix} b_1 \\ -2b_3 + 2b_5 \\ b_3 \\ -2b_5 \\ b_5 \end{bmatrix} = b_1\begin{bmatrix} 1 \\ 0 \\ 0 \\ 0 \\ 0 \end{bmatrix} + b_3\begin{bmatrix} 0 \\ -2 \\ 1 \\ 0 \\ 0 \end{bmatrix} + b_5\begin{bmatrix} 0 \\ 2 \\ 0 \\ -2 \\ 1 \end{bmatrix}$$  
So, the solution space is of dimension 3. The solution space is spanned by 
\[ \left\{\begin{bmatrix}
1 \\ 0 \\ 0 \\ 0 \\ 0 
\end{bmatrix}, \ \ \begin{bmatrix}
0 \\ -2 \\ 1 \\ 0 \\ 0
\end{bmatrix}, \ \ \begin{bmatrix}
0 \\ 2 \\ 0 \\ -2 \\ 1
\end{bmatrix} \right\}.
\]
We can choose three independent vectors from the solution space to form the 3 rows of matrix $B$.
For example, we can let
\[ B = \begin{bmatrix}
1 & 0 & 0 & 0 & 0 \\
0 & -2 & 1 & 0 & 0 \\
0 & 2 & 0 & -2 & 1
\end{bmatrix}. \]
For each row $\vec{b}_i^T, i = 1, 2, 3$ of $B$, it is easy to check that $\begin{bmatrix}
\vec{r}_1^T \\ \vec{r}_2^T
\end{bmatrix} \vec{b}_i = 0$, which implies that $B\vec{r}_1 = B\vec{r}_2 = 0$. And $B$ has rank 3 because the $1^{\text{st}}$, $3^{\text{rd}}$ and last column of $B$ makes the basis for $\R^3$. Thus the null space of $B$ is spanned by
$\vec{r}_1,\ \vec{r}_2$, which is identical to the row space of $A$.
\end{enumerate}
}
\end{homeworkProblem}


% Problem 2
\begin{homeworkProblem}
Let $A$ be an $m \x n$ matrix with rank $r \leq \min\{m,n\}$. Depending on $m,\ n \tand r$, a system $A\vec x=\vec b$ can have none, one, or infinitely many solutions.\\
For what choices of $m,\ n \tand r$ do each of the following cases hold? If no such $m,\ n \tand r$ can be found explain why not.
\begin{enumerate}[(a)]
\item $A\vec{x}=\vec{b}$ has no solutions, regardless of $\vec{b}$
\item $A\vec{x}=\vec{b}$ has exactly 1 solution for any $\vec{b}$
\item $A\vec{x}=\vec{b}$ has infinitely many solutions for any $\vec{b}$
%\item $A\vec{x}=\vec{b}$ has no {\bf or} just 1 solution depending on $\vec{b}$
%\item $A\vec{x}=\vec{b}$ has no {\bf or} infinitely many solutions depending on $\vec{b}$
%\item $A\vec{x}=\vec{b}$ has only 1 {\bf or} infinitely many solutions depending on $\vec{b}$
\end{enumerate}
({\bf Hint}: think of what conditions the column vectors and column space of $A$ should satisfy.)

\problemAnswer{
\begin{enumerate}[(a)]
\item $A\vec{x}=\vec{b}$ has no solutions, regardless of $\vec{b}$\\
{\bf Answer:} Impossible. Since for $\vec{b} = 0$ we always have a solution $\vec{x} = 0$.

\item $A\vec{x}=\vec{b}$ has exactly 1 solution for any $\vec{b}$\\
{\bf Answer:} For any $\vec{b} \in \R^m$ there is exactly one solution to $A\vec{x} = \vec{b}$, this happens if and only if the columns of $A$ form a basis of space $\R^m$. That is, if and only if the column space of $A$ is the full space $\R^m$ and the columns are independent. That is, if and only if $m = n = r$. (Hence if and only if $A$ is square and nonsingular.)

\item $A\vec{x}=\vec{b}$ has infinitely many solutions for any $\vec{b}$\\
{\bf Answer:} For any $\vec{b} \in \R^m$ there are infinite solutions to $A\vec{x} = \vec{b}$. That is, for any $\vec{b} \in \R^m$ we can find a solution to $A\vec{x} = \vec{b}$ and add to this solution any element of the nullspace of $A$. Therefore we need the nullspace of $A$ to have an infinite number of vectors in it. In other words, the nullspace dimension should be at least 1, i.e. $n-r \ge 1$. Since there is at least one solution for any $\vec{b}$, the columns
of $A$ must span the full space $\R^m$. Note that the columns of $A$ live in $\R^m$. So we need $r = m$. \\
On the other hand, if the column space of $A$ is the full space $(r = m)$ and the
columns are dependent $(r < n)$, then for any $\vec{b} \in \R^m$ there are infinite solutions to $A\vec{x} = \vec{b}$. So the converse is also true.\\
Therefore a necessary and sufficient condition is that the column space of $A$ is the full
space $(m = r)$ and the columns are dependent $(r < n)$. That is, the statement is true if and only if $m = r < n$.

%\item $A\vec{x}=\vec{b}$ has no {\bf or} just 1 solution depending on $\vec{b}$\\
%{\bf Answer:} If and only if the column space of $A$ is not the full space $\R^m$, i.e. $r < m$, and the columns are independent. That is, if and only if $n = r < m$. This condition is necessary and sufficient because we have: (i) ``There exits $\vec{b}_1$ such that $A\vec{x} = \vec{b}_1$ does not have a solution." $\iff$ ``The column space of A is not the full space $\R^m$."; (ii) ``There exits $\vec{b}_2$ such that $A\vec{x} = \vec{b}_2$ has exactly one solution." $\iff$ ``The columns of $A$ are independent."
%
%\item $A\vec{x}=\vec{b}$ has no {\bf or} infinitely many solutions depending on $\vec{b}$\\
%{\bf Answer:} If and only if the column space of $A$ is not the full space $\R^m$, i.e. $r < m$, and the columns are dependent. That is, if and only if $r < m \tand r < n$. This condition is necessary and sufficient because we have: (i) ``There exits $\vec{b}_1$ such that $A\vec{x} = \vec{b}_1$ does not have a solution." $\iff$ ``The column space of $A$ is not the full space $\R^m$."; (ii) ``There exits $\vec{b}_2$ such that $A\vec{x} = \vec{b}_2$ has infinitely many solutions." $\iff$ ``The columns of $A$ are dependent."
%
%\item $A\vec{x}=\vec{b}$ has only 1 {\bf or} infinitely many solutions depending on $\vec{b}$\\
%{\bf Answer:} Impossible. Note that if there exits $\vec{b}_1$ such that the number of solutions to $A\vec{x} = \vec{b}_1$ equals 1, this implies that the columns of $A$ are independent. On the other hand, if there exits $\vec{b}_2$ such that the number of solutions to $A\vec{x} = \vec{b}_2$ equals $\infty$, this implies that the columns of $A$ are dependent. They contradict each other.
\end{enumerate}

{\bf Remark}\\
Note that the equation $A\vec{x} = \vec{b}$ could have 
\begin{itemize}
\item exactly 1 solution for any $\vec{b}$ (part (b))
\item infinity many solutions for any $\vec{b}$ (part (c))
\item no solution for some $\vec{b}$ and has 1 solution for some $\vec{b'}$
\item no solution for some $\vec{b}$ and has infinitely many solutions for some $\vec{b'}$
\end{itemize}
Exercise: Think about the conditions that make the $3^{\text{rd}} \tand 4^{\text{th}}$ cases true.
}
\end{homeworkProblem}

% Problem 3
\begin{homeworkProblem}
Consider a matrix product $AB$, where $A$ is $m\x n \tand B$ is $n\x p$. Show that the column space of $AB$ is contained in the column space of $A$. Give an example of matrices $A,\ B$ such that those two spaces are not identical.\\
{\bf Definition.} A vector space $U$ is {\it contained} in another vector space $V$ (denoted as $U \subseteq V$) if every vector $\vec u \in U$ ($\vec u$ in vector space $U$) is also in $V$.
\\
{\bf Definition.} We say that two vector spaces are {\it identical} (equal) if $U \subseteq V$ {\bf and} $V \subseteq U$. \\
(e.g. $V$ is identical to itself since $V \subseteq V$ and $V \subseteq V$.)

\problemAnswer{
Let $B = [\vec{b}_1, \vec{b}_2, \ldots, \vec{b}_p]$ and $A = [\vec{a}_1, \vec{a}_2, \ldots, \vec{a}_n]$. We have to show that $\mathcal{R}(AB) \subseteq \mathcal{R}(A)$. Since the column space of $AB$ is the span of columns of $AB$ and column space of $A$ is the span of columns of $A$, it suffices to show that any column $j = 1, 2, \ldots, p$ of $AB$ is in $\mathcal{R}(A)$. Consider that any column $j$ of $AB$ is $A\vec{b}_j$. We have to show that $A\vec{b}_j$ can be written as a linear combination of columns of $A$. This is indeed the case since any column $j$ of $AB$ is $$A\vec{b}_j = [\vec{a}_1, \vec{a}_2, \ldots, \vec{a}_n]\vec{b}_j = \sum_{k=1}^n b_{kj}\vec{a}_k,$$ which is a linear combination of columns of $A$. So $A\vec{b}_j \in \mathcal{R}(A)$ for all $j=1,\ldots,p$. Since every column of $AB$ is in $\mathcal{R}(A)$, so $\mathcal{R}(AB) \subseteq \mathcal{R}(A)$. \\ \\
In order to find an example, we consider that if the two spaces are not identical, then the column space of $AB$ can only be strictly smaller than the column space of $A$. So when trying to construct examples, we want $\rank(AB) < \rank(A)$. 
\\For example, we can take $A$ to be $m \times m$ and full rank, while $B$ is $m \times p$ with $p < m$. Since $B$ has only $p$ columns we know that $\rank(AB) < \rank(A)$.\\
As another example: 
\[ A = \begin{bmatrix}
1 & 0 & 0 \\
0 & 1 & 0 \\
0 & 0 & 1 
\end{bmatrix}, \ \ 
B = \begin{bmatrix}
1 & 1 & 0 \\
0 & 1 & 1 \\
0 & 0 & 0 
\end{bmatrix}.\]
}
\end{homeworkProblem}

% Problem 4
\begin{homeworkProblem}
An $n \x n$ matrix $A$ has a property that the elements in each of its rows sum to 1. Let $P$ be any $n \x n$ permutation matrix. Prove that $(P-A)$ is singular.

\problemAnswer{
$(P-A)$ is a matrix such that the elements in each of its rows sum to 0. This is because the permutation matrix $P$ has exactly one entry 1 in each row and 0's elsewhere. 

Consider that $(P-A)$ is singular if any only if $\mathcal{N}(P-A)$ contains a nonzero vector. Recall that $\mathcal{N}(P-A) = \left\{\vec{x} \in \R^n | (P-A)\vec{x} = \vec{0} \right\}$. Therefore, in order to show that $\mathcal{N}(P-A)$ contains at least one nonzero vector, we have to find some vector $\vec{v} \neq \vec{0}$ such that $(P-A)\vec{v} = \vec{0}$.

We explore the fact that the elements in each of the rows of $(P-A)$ sum to 0. Let $c_{ij}$ be the $(i,j)$-th entry of matrix $(P-A)$. Then we have 
\[\vec{0} = \begin{bmatrix}\sum_{j=1}^n c_{1j} \\ \sum_{j=1}^n c_{2j} \\ \vdots \\ \sum_{j=1}^n c_{nj} \end{bmatrix} = (P-A)\begin{bmatrix} 1 \\ 1 \\ \vdots \\ 1 \end{bmatrix}
\]
Now let $\vec{v} = [1, 1, \ldots, 1]^T$ be a vector with 1s in all its entries. Then $(P-A)\vec{v} = \vec 0$, which means that $\vec{v}$ is in the null space of $(P-A)$. Since $\vec{v} \neq \vec{0}$, we found a nonzero vector in $\mathcal{N}(P-A)$, that is, $\text{dim } \mathcal{N}(P-A) > 0$. Thus, $(P-A)$ is singular. 
}
\end{homeworkProblem}

%Problem 5
\begin{homeworkProblem}
Let $V \tand W$ be 3 dimensional subspaces of $\R^5$. Show that $V \tand W$ must have at least one nonzero vector in common.

\problemAnswer{
\textbf{Solution I}

Let bases of $V \tand W$ be $\{\vec{v}_1, \vec{v}_2, \vec{v}_3\} \tand \{\vec{w}_1, \vec{w}_2, \vec{w}_3\}$, respectively. Then span$\{\vec{v}_1, \vec{v}_2, \vec{v}_3, \vec{w}_1, \vec{w}_2, \vec{w}_3\}$ is a subspace of $\R^5$. Since the dimension of $\R^5$ is 5, the six vectors $\vec{v}_1, \vec{v}_2, \vec{v}_3, \vec{w}_1, \vec{w}_2, \vec{w}_3$ cannot be independent. 
Thus, there exist constants $\alpha_i, \beta_i,\ i = 1,2,3$ (these constants cannot all be zeros) such that $$\alpha_1\vec{v}_1 + \alpha_2\vec{v}_2 + \alpha_3\vec{v}_3 + \beta_1\vec{w}_1 + \beta_2\vec{w}_2 + \beta_3\vec{w}_3 = \vec 0 \in \R^5.$$ 
Rearranging the terms, we can write 
\[ \vec{u} = \alpha_1\vec{v}_1 + \alpha_2\vec{v}_2 + \alpha_3\vec{v}_3 = -(\beta_1\vec{w}_1 + \beta_2\vec{w}_2 + \beta_3\vec{w}_3) \in \R^5. \]
If $\vec{u}$ is zero, we must have $\alpha_i = 0,\ i = 1,2,3$, since $\{\vec{v}_1, \vec{v}_2, \vec{v}_3\}$ is independent, and $\beta_i = 0,\ i = 1,2,3$, since $\{\vec{w}_1, \vec{w}_2, \vec{w}_3\}$ is independent. However, by the argument above, we cannot have all the constants to be zeros, thus $\vec{u} \neq 0$. 

We have $\vec{u} \in V$, since $\vec{u} = \alpha_1\vec{v}_1 + \alpha_2\vec{v}_2 + \alpha_3\vec{v}_3$. Similarly $\vec{u} \in W$, since $\vec{u} = -(\beta_1\vec{w}_1 + \beta_2\vec{w}_2 + \beta_3\vec{w}_3)$. Thus, we have a nonzero vector $\vec{u} \in \R^5$, which is both in $V$ and in $W$. 
}

\problemAnswer{
\textbf{Solution II}

We use proof by contradiction. Let bases of $V$ and $W$ be $\{\vec{v}_1, \vec{v}_2, \vec{v}_3\}$ and $\{\vec{w}_1, \vec{w}_2, \vec{w}_3\}$, respectively. We assume that $V$ and $W$ are disjoint. Then any linear combination of $\{\vec{v}_1, \vec{v}_2, \vec{v}_3\}$ cannot be expressed as a linear combination of $\{\vec{w}_1, \vec{w}_2, \vec{w}_3\}$. Equivalently, the equation $$c_1\vec{v}_1 + c_2\vec{v}_2 + c_3\vec{v}_3 + c_4\vec{w}_1 + c_5\vec{w}_2 + c_6\vec{w}_3 = 0$$ can only have a trivial solution $\vec{c} = [c_1, c_2, c_3, c_4, c_5, c_6]^T = \vec 0$. 

Let $A = [\vec{v}_1, \vec{v}_2, \vec{v}_3, \vec{w}_1, \vec{w}_2, \vec{w}_3]$. So $A$ is a $5 \x 6$ matrix. The linear equation $A\vec{x} = \vec 0$ is a homogeneous underdetermined system, so it must have a non-trivial solution. This contradiction proves that $V$ and $W$ cannot be disjoint. 
} 
\end{homeworkProblem}

% Problem 6
\begin{homeworkProblem}
\begin{enumerate}[(a)]
%(5)
\item The nonzero column vectors $\vec{u} \tand \vec{v}$ have $n$ elements. An $n \x n$ matrix $A$ is given by $A = \vec{u}\vec{v}^T$ ({\bf Note:} this is different from the innerproduct (also sometimes known as the dot product), which we would write as $\vec{v}^T\vec{u}$). Show that the rank of $A$ is 1.

 %(10)
\item Show that the converse is true. That is, if the rank of a matrix $A$ is 1, then we
can find two vectors $\vec{u} \tand \vec{v}$, such that $A = \vec{u}\vec{v}^T$.
\end{enumerate}

\problemAnswer{
\begin{enumerate}[(a)]
\item Since $\vec{u}$ is nonzero, suppose $u_i \neq 0$ for some $i$. Similarly suppose $v_j \neq 0$ for some $j$. Then the $(i,j)$-th element of matrix $A$ is $a_{i,j} = u_iv_j \neq 0$. Therefore, the rank of $A$ is at least one: $\text{r}(A) \geq 1$.

On the other hand, from Problem 3 we know that the column space of $A = \vec{u}\vec{v}^T$ is contained in the column space of $\vec{u}$, which has at most one dimension since $\vec u$ is a vector. Thus, $\text{r}(A) \leq 1$.

Therefore, we have $\text{r}(A) = 1$. 

\item To show the converse is true, observe that since the column space of $A$ is of dimension one, its basis has only one vector. Let this vector be $\vec{u}$. Then every column of $A$ is in the space spanned by the vector $\vec{u}$. That is, for every $i = 1, 2, \ldots, n$, we can find a real number $v_i \in \R$, such that $\vec{a}_i = v_i\vec{u}$. Then we have $$A = [\vec{a}_1, \vec{a}_2, \ldots, \vec{a}_n] = [v_1\vec{u}, v_2\vec{u}, \ldots, v_n\vec{u}] = \vec{u}[v_1, v_2, \ldots, v_n] = \vec{u}\vec{v}^T.$$
\end{enumerate}
}
\end{homeworkProblem}


\end{document}
