%%%%%%%%%%%%%%%%%%%%%%%%%%%%%%%%%%%%%%%%%
% Structured General Purpose Assignment
% LaTeX Template
%
% This template has been downloaded from:
% http://www.latextemplates.com
%
% Original author:
% Ted Pavlic (http://www.tedpavlic.com)
%
% Note:
% The \lipsum[#] commands throughout this template generate dummy text
% to fill the template out. These commands should all be removed when 
% writing assignment content.
%
%%%%%%%%%%%%%%%%%%%%%%%%%%%%%%%%%%%%%%%%%

%----------------------------------------------------------------------------------------
%	PACKAGES AND OTHER DOCUMENT CONFIGURATIONS
%----------------------------------------------------------------------------------------

\documentclass{article}

\usepackage{fancyhdr} % Required for custom headers
\usepackage{extramarks} % Required for headers and footers
\usepackage{graphicx} % Required to insert images
\usepackage{enumerate}

% Margins
\topmargin=-0.45in
\evensidemargin=0in
\oddsidemargin=0in
\textwidth=6.5in
\textheight=9.0in
\headsep=0.25in 

\linespread{1.1} % Line spacing

% Set up the header and footer
\pagestyle{fancy}
\lhead{Linear Algebra with Application\\
to Engineering Computation}
\chead{}
\rhead{CME 200/ME300A\\
M. Gerritsen\\
Fall 2013}
\headheight = 40pt

\renewcommand\headrulewidth{0.4pt} % Size of the header rule
\renewcommand\footrulewidth{0.4pt} % Size of the footer rule

\setlength\parindent{0pt} % Removes all indentation from paragraphs

%----------------------------------------------------------------------------------------
%	DOCUMENT STRUCTURE COMMANDS
%	Skip this unless you know what you're doing
%----------------------------------------------------------------------------------------

% Header and footer for when a page split occurs within a problem environment
\newcommand{\enterProblemHeader}[1]{
\nobreak\extramarks{#1}{#1 continued on next page\ldots}\nobreak
\nobreak\extramarks{#1 (continued)}{#1 continued on next page\ldots}\nobreak
}

% Header and footer for when a page split occurs between problem environments
\newcommand{\exitProblemHeader}[1]{
\nobreak\extramarks{#1 (continued)}{#1 continued on next page\ldots}\nobreak
\nobreak\extramarks{#1}{}\nobreak
}

\setcounter{secnumdepth}{0} % Removes default section numbers
\newcounter{homeworkProblemCounter} % Creates a counter to keep track of the number of problems

\newcommand{\homeworkProblemName}{}
\newenvironment{homeworkProblem}[1][Problem \arabic{homeworkProblemCounter}]{ % Makes a new environment called homeworkProblem which takes 1 argument (custom name) but the default is "Problem #"
\stepcounter{homeworkProblemCounter} % Increase counter for number of problems
\renewcommand{\homeworkProblemName}{#1} % Assign \homeworkProblemName the name of the problem
\section{\homeworkProblemName} % Make a section in the document with the custom problem count
\enterProblemHeader{\homeworkProblemName} % Header and footer within the environment
}{
\exitProblemHeader{\homeworkProblemName} % Header and footer after the environment
}

\title{Assignment 1 - Background review}
\date{Issued: \today}
\author{Due: October 2, in class\\
No late assignments accepted}

%----------------------------------------------------------------------------------------

\begin{document}
\maketitle
\thispagestyle{fancy}
\textbf{Important:}
\begin{itemize}
\item Give complete answers: Do not only give mathematical formulae, but explain what you are doing. Conversely, do not leave out critical intermediate steps in mathematical derivations. Point values are in parentheses.
\item Write your \textbf{name} as well as your \textbf{Sunet ID} on your assignment. \textbf{Please staple pages together.} Points will be docked otherwise.
\item Questions preceded by * are harder and/or more involved.
\end{itemize}

\begin{homeworkProblem}[Problem \arabic{homeworkProblemCounter}  (10)]
Indicate whether the following statements are TRUE or FALSE and motivate your answers clearly. To show a statement is false, it is sufficient to give one counterexample. If a statement is true, provide a general proof.
\begin{enumerate}[(a)]
\item If $A^2 + A = I$ then $A^{-1} = I + A $
\item If all diagonal entries of A are zero, then A is singular.
\end{enumerate}
\end{homeworkProblem}

\begin{homeworkProblem}[Problem \arabic{homeworkProblemCounter}  (10)]
The product of two $n \times n$ lower triangular matrices is again lower triangular (all its entries above the main diagonal are zero). Prove it in general and confirm this with a 3-by-3 example. 
\end{homeworkProblem}

\begin{homeworkProblem}[Problem \arabic{homeworkProblemCounter}  (10)]
If $A = A^T$ and $B = B^T$ , which of these matrices are certainly symmetric? Justify your answer.
\begin{enumerate}[(a)]
\item $A^2 - B^2$
\item $(A + B)(A - B)$
\item $ABA$
\item $ABAB$
\end{enumerate}
\end{homeworkProblem}

\begin{homeworkProblem}[Problem \arabic{homeworkProblemCounter}  (10)]
A {\it skew-symmetric} matrix is a matrix that satisfies $A^T = -A$. Prove that if $A$ is a skew-symmetric matrix, then for any vector $x$, we must have $x^TAx = 0$.
\end{homeworkProblem}

\begin{homeworkProblem}[Problem \arabic{homeworkProblemCounter}  (10)]
Suppose $A$ is an invertible matrix, and you exchange its first two rows to create a new matrix $B$.  Is the new matrix $B$ necessarily invertible?  If so, how could you find $B^{-1}$ from $A^{-1}$?  If not, why not?
\end{homeworkProblem}

\begin{homeworkProblem}[Problem \arabic{homeworkProblemCounter}  (10)]
Let $A$ be an invertible $n\times n$ matrix.  Prove that $A^m$ is also invertible and that $$\left(A^m\right)^{-1} = \left(A^{-1}\right)^m$$ for $m= 1, 2, 3,  \ldots$ 
\end{homeworkProblem}



\begin{homeworkProblem}[Problem \arabic{homeworkProblemCounter}  (10)]
Let $A$ be a $2\times 2$ matrix $\left( \begin{array}{cc}
a_{11} & a_{12} \\
a_{21} & a_{22} \end{array} \right)$ with $a_{11} \neq 0$ and let $\alpha = a_{21}/a_{11}$.  Show that $A$ can be factored into a product of the form
\[ \left( \begin{array}{cc}
1 & 0 \\
\alpha & 1 \end{array} \right) 
\left( \begin{array}{cc}
a_ {11} & a_{12}\\
0 & b  \end{array} \right)\]

What is the value of $b$?

\end{homeworkProblem}

\begin{homeworkProblem}[Problem \arabic{homeworkProblemCounter}*  (20: 10 points each part)]
\begin{enumerate}[(a)]
\item Show that if the $n\times n$ matrices $A$ and $B$ are invertible, and if the matrix $A+B$ is also invertible, then the matrix $B^{-1} + A^{-1}$ is also invertible. 
\item Assume that $C$ is a skew-symmetric matrix and that $D$ is a matrix defined as $$D=(I+C)(I-C)^{-1}$$  Prove that $D^TD=DD^T=I.$
\end{enumerate}
\end{homeworkProblem}

\begin{homeworkProblem}[Problem \arabic{homeworkProblemCounter}*  (10)]
Given an $n \times n$ matrix $A$ with column vectors $\vec{a_1},\vec{a_2},\ldots,\vec{a_n},$ construct a matrix $B$ such that the matrix $AB$ has the columns $\vec{u_1},\vec{u_2},\ldots,\vec{u_n}$ with the following properties:
\begin{enumerate}[(i)]
\item $\vec{u_i} = \vec{a_i}, \quad i\neq j$
\item $\vec{u_j} = \sum_{k=1}^j\alpha_k\vec{a_k}$,
\end{enumerate}
where $j$ is a fixed integer with $1\leq j \leq n$ and $j\neq 0$. 

\end{homeworkProblem}

%\begin{homeworkProblem}
%\textbf{BONUS}\\
%It is a well-known theorem that the set of all n by n invertible matrices form a group under matrix multiplication. Here you will prove this result for the set of all n by n invertible diagonal matrices. Recall that a group in the sense of abstract algebra is defined as a set $G$ equipped with a binary operation $\odot$ such that it exhibits
%\begin{itemize}
%\item \textbf{Closure:} $\forall a, b \in G, a\odot b \in G$
%\item \textbf{Associativity:} $\forall a, b, c \in G, (a\odot b)\odot c = a\odot (b\odot c)$
%\item \textbf{Identity Element:} $\exists e \in G \mid e\odot a = a\odot e = a$
%\item \textbf{Inverse Element:} $\forall a \in G, \exists b \in G \mid a\odot b = b\odot a = e$
%\end{itemize}
%Show that the set of n by n invertible diagonal matrices satisfy the axioms of a group under multiplication.
%\emph{HINT:} Try out a few examples on 3 by 3 matrices and generalize. 
%\end{homeworkProblem}

\end{document}
