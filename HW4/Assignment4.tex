%

\documentclass{article}

\usepackage{fancyhdr} % Required for custom headers
\usepackage{extramarks} % Required for headers and footers
\usepackage{graphicx} % Required to insert images
\usepackage{enumerate}
\usepackage{amsmath}
\usepackage{bbold}

% Margins
\topmargin=-0.45in
\evensidemargin=0in
\oddsidemargin=0in
\textwidth=6.5in
\textheight=9.0in
\headsep=0.25in 

\linespread{1.1} % Line spacing

% Set up the header and footer
\pagestyle{fancy}
\lhead{Linear Algebra with Application\\
to Engineering Computation}
\chead{}
\rhead{CME 200/ME300A\\
M. Gerritsen\\
Fall 2013}
\headheight = 20pt



%th in the exponent (e.g. when writing ith, instead use i$\eth$)
\newcommand{\eth}{^{\text{th}}}


\newcommand{\R}{\mathbb{R}}

%short-cuts for Greek letters
\newcommand{\al}{\alpha}
\newcommand{\dlt}{\delta}
\newcommand{\eps}{\epsilon}

%times (cross-product)
\newcommand{\x}{\times}
%inverse
\newcommand{\inv}{^{-1}}
%cond
\newcommand{\cond}{\mathrm{cond}}
%trace
\newcommand{\trace}{\mathrm{trace}}

\newcommand{\twith}{\text{ with }}
\newcommand{\tand}{\text{ and }}
\newcommand{\tfor}{\text{ for }}
\newcommand{\tor}{\text{ or }}

\newcommand{\ip}{_{i+1}}
\newcommand{\im}{_{i-1}}

\newcommand{\half}{\frac{1}{2}}
\newcommand{\oneby}[1]{\frac{1}{#1}}
\newcommand{\overto}[1]{\overset{#1}{\longrightarrow}} 
 
\renewcommand\headrulewidth{0.4pt} % Size of the header rule
\renewcommand\footrulewidth{0.4pt} % Size of the footer rule

\setlength\parindent{0pt} % Removes all indentation from paragraphs

%----------------------------------------------------------------------------------------
%       DOCUMENT STRUCTURE COMMANDS
%       Skip this unless you know what you're doing
%----------------------------------------------------------------------------------------

% Header and footer for when a page split occurs within a problem environment
\newcommand{\enterProblemHeader}[1]{
\nobreak\extramarks{#1}{#1 continued on next page\ldots}\nobreak
\nobreak\extramarks{#1 (continued)}{#1 continued on next page\ldots}\nobreak
}

% Header and footer for when a page split occurs between problem environments
\newcommand{\exitProblemHeader}[1]{
\nobreak\extramarks{#1 (continued)}{#1 continued on next page\ldots}\nobreak
\nobreak\extramarks{#1}{}\nobreak
}

\setcounter{secnumdepth}{0} % Removes default section numbers
\newcounter{homeworkProblemCounter} % Creates a counter to keep track of the number of problems

\newcommand{\homeworkProblemName}{}
\newenvironment{homeworkProblem}[1][Problem \arabic{homeworkProblemCounter}]{ % Makes a new environment called homeworkProblem which takes 1 argument (custom name) but the default is "Problem #"
\stepcounter{homeworkProblemCounter} % Increase counter for number of problems
\renewcommand{\homeworkProblemName}{#1} % Assign \homeworkProblemName the name of the problem
\section{\homeworkProblemName} % Make a section in the document with the custom problem count
\enterProblemHeader{\homeworkProblemName} % Header and footer within the environment
}{
\exitProblemHeader{\homeworkProblemName} % Header and footer after the environment
}
\newcommand\overmat[2]{%
  \makebox[0pt][l]{$\smash{\overbrace{\phantom{%
    \begin{matrix}#2\end{matrix}}}^{\text{$#1$}}}$}#2}

\newcommand{\problemAnswer}[1]{ % Defines the problem answer command with the content as the only argument
\noindent\framebox[\columnwidth][c]{\begin{minipage}{0.98\columnwidth}#1\end{minipage}} % Makes the box around the problem answer and puts the content inside
}

\title{Assignment 4}
\date{Issued: October 16, 2013}
\author{Due: October 23, in class\\
No late assignments accepted}

%----------------------------------------------------------------------------------------

\begin{document}

\maketitle
\thispagestyle{fancy}
\textbf{Important:}
\begin{itemize}
\item Give complete answers: Do not only give mathematical formulae, but explain what you are doing. Conversely, do not leave out critical intermediate steps in mathematical derivations.
\item Write your \textbf{name} as well as your \textbf{Sunet ID} on your assignment. \textbf{Please staple pages together.}
\item Questions preceded by  $\star$  are harder and/or more involved.
\item \textbf{Include code with your assignment.}
\item Comment any graphs and plots on the same page as the graph or plot itself.
\end{itemize}

% Problem 1
\begin{homeworkProblem} %(15)
In assignment 2, we discretized the 1-dimensional heat equation with Dirichlet boundary conditions:
\begin{align*}
\frac{d^2T}{dx^2} &= 0,0\leq x\leq 1 \\
T(0) &= 0,T(1) = 2
\end{align*}
The discretization leads to the matrix-vector equation $At = b$, with
\begin{align}
A = 
\begin{pmatrix}
-2 &  1  & 0 & \ldots & 0\\
1  &  -2 & 1 & \ldots & 0\\
\vdots & \ddots & \ddots & \ddots & \vdots\\
0   &\ldots & 1 & -2 & 1\\
0   &\ldots & 0 & 1 & -2
\end{pmatrix}, b =
\begin{pmatrix}
0\\
0\\
\vdots\\
0 \\
-2
\end{pmatrix}
\end{align}
Here $A$ is an $(N-1) \times (N-1)$ matrix.
\begin{enumerate}[a.]
\item
(10 pts) Find the LU factorization of $A$ for $N = 10$ using \texttt{Matlab}. Is \texttt{Matlab} using any pivoting to find the LU decomposition? Find the inverse
of $A$ also. As you can see the inverse of $A$ is a dense matrix.
\textbf{Note:} The attractive sparsity of $A$ has been lost when computing its inverse, but $L$ and $U$ are sparse. Generally speaking, banded matrices have $L$ and $U$ with similar band structure. Naturally we then prefer to use the $L$ and $U$ matrices to compute the solution, and not the inverse. Finding $L$ and $U$ matrices with least "fill-in" ("fill-in" refers to nonzeros appearing at locations in the matrix where $A$ has a zero element) is an active research area, and generally involves sophisticated matrix re-ordering algorithms.
\item
(10 pts) Compute the determinants of $L$ and $U$ for $N = 1000$ using \texttt{Matlab}'s determinant command. Why does the fact that they are both nonzero imply that $A$ is non-singular? How could you have computed these determinants really quickly yourself without \texttt{Matlab}'s determinant command?

\end{enumerate}
\end{homeworkProblem}

  
% Problem 2
\begin{homeworkProblem} %(15)
\begin{enumerate}[a.]
\item
\begin{enumerate}[(i)]
\item
(10 pts) Use \texttt{Matlab} command \texttt{A=rand(4)} to generate a random 4-by-4 matrix and then use the function \texttt{qr} to find an orthogonal matrix $Q$ and an upper triangular matrix $R$ such that $A = QR$. Compute the determinants of $A$, $Q$ and $R$.
\item
(15 pts) Set \texttt{A=rand(n)} for at least 5 different $n$'s in \texttt{Matlab} for computing the determinant of $Q$ where $Q$ is the orthogonal matrix generated by \texttt{qr(A)}. What do you observe about the determinants of the matrices $Q$? Show, with a mathematical proof, that the determinant of any orthogonal matrix is either 1 or -1.
\item
(10 pts) For a square $n \times n$ matrix matrix $B$, suppose there is an orthogonal matrix $Q$ and an upper-triangular matrix $R$ such that $B=QR$. Show that if a vector $x$ is a linear combination of the first $k$ column vectors of $B$ with $k\leq n$, then it can also be expressed as a linear combination of the first $k$ columns of $Q$.
\end{enumerate}
\item[b*.]
\begin{enumerate}[(i)]
\item 
(10 pts) Assume $\{\vec{v_1},\vec{v_2} \cdots \vec{v_n} \}$ is an orthonormal basis of $\mathbb{R}^n$. Suppose there exists a unit vector $\vec{u}$ such that $\vec{u}^T \vec{v_k} = 0$ for all $k=2,3\cdots n$, show that $\vec{u} = \vec{v_1}$ or $\vec{u} = -    \vec{v_1}$.
\item
(10 pts) Prove that if $C$ is non-singular and $C = QR$, where $Q$ is an orthogonal matrix and $R$ is an upper-triangular matrix with diagonal elements all positive, then the $Q$ and $R$ are unique. \\
\textbf{Hint:} Use proof by contradiction.
\end{enumerate}
\end{enumerate}
\end{homeworkProblem}
  
% Problem 3
\begin{homeworkProblem} %(15)
In class we have introduced the $LU$ decomposition of $A$, where $L$ is unit-lower triangular, in that it has ones along the diagonal, and $U$ is upper triangular. However, in the case of symmetric matrices, such as the discretization matrix, it is possible to decompose $A$ as $LDL^T$, where $L$ is still unit-lower triangular and $D$ is diagonal. This decomposition clearly shows off the symmetric nature
of $A$. 
\begin{enumerate}[a.]
\item
 (10 pts) Find the $LDL^T$ decomposition for
the matrix given in \textbf{Problem 1}. Show that $L$ is bidiagonal.
How do $D$ and $L$ relate to the matrix $U$ in the $LU$ decomposition of $A$? \\
\textbf{Hint:} Think about how $D$ and $L^T$ relate to $U$.\\
\textbf{Note:} Computing $LDL^T$ this way does not work out for any symmetric matrix, it only happens to work for this matrix in particular.
\item[b*.]
\begin{enumerate}[(i)]
\item
 (10 pts) To solve $A\vec{x} = \vec{b}$  we can exploit this new decomposition. We get $L D L^T \vec{x} = \vec{b}$
which we can now break into three parts: Solve $L\vec{y} = \vec{b}$ using forward substitution, now solve $D \vec{z} = \vec{y}$, and then solve $L^T \vec{x} = \vec{z}$ using back substitution. Write a \texttt{Matlab} code that does exactly this for arbitrary $N$ for the $A$ in \textbf{Problem 1}.
\item
(5 pts) Solve a system of the same form as \textbf{Problem 1} for $A$ of size $10$ and of size $1000$ with $\vec{b}$ having all zeros except 2 as the last entry in both cases, and verify the correctness of your solution using \texttt{Matlab}'s \texttt{A\textbackslash b} operator and the \texttt{norm} command.
\end{enumerate}
 \end{enumerate}
\end{homeworkProblem} 
\end{document}